% University of Alberta Example Thesis
% By the Rogue's Gallery, Department of Computing Science
% Last updated 16 Dec 2004

% Note: you will probably have to edit the thesis.sty file to comment or uncomment bits and pieces (e.g. co-supervisor, externals, etc)

\documentclass[10pt]{report}          % for default format

%\usepackage{doublespace}
\usepackage{graphics}
\graphicspath{./pdf/} 
\usepackage{epsf}
\usepackage{longtable}
%% For bar graphs
%\usepackage{bar}
\usepackage{rotating}
\usepackage{algorithm}
\usepackage{algorithmic}
%\usepackage[first,bottomafter]{draftcopy}
\usepackage{times}
\usepackage{thesis}

%%% Paul's favourite macros
\newcommand{\myem}[1]{{\em{#1}\/}}
%-- Reference Chapter
\newcommand{\refChapter}[1]{Chapter~\ref{#1}}
%-- Reference Section
\newcommand{\refSection}[1]{Section~\ref{#1}}
%-- Reference Table
\newcommand{\refTable}[1]{Table~\ref{#1}} 
%-- Reference Figure
\newcommand{\refFigure}[1]{Figure~\ref{#1}}
%-- in-line code fragment
\newcommand{\C}[1]{\begin{normalsize}\begin{tt}#1\end{tt}\end{normalsize}}
%% Paul added this
%% For extended functionality of the tabular environment
%%      (see Latex companion pg. 108 in particular)
\usepackage{array}
%% Paul added this
\newcommand{\PreserveBackslash}[1]{\let\temp=\\#1\let\\=\temp}
\let\PBS=\PreserveBackslash  % shorthand

\newcommand{\myColRaggedright}{\PBS\raggedright\hspace{0pt}}
\newcommand{\myColRaggedleft}{\PBS\raggedleft\hspace{0pt}}

%% Correct title for TOC
\renewcommand{\contentsname}{Table of Contents}

\title{Using Combined Profile to Decide When Thread Level Speculation is Profitable}
\author{Arnamoy Bhattacharyya}
\degree{Master of Science} % or "Doctor of Philosophy", this string appears
                                        % as written in the formated thesis.
\dept{Computing Science}  % Write Computing Science or Civil Engineering.

\supervisor{Jos\'{e} Nelson Amaral, Computing Science} % Add university if not from University of Alberta
%\cosupervisor{Dale Schuurmans, Computing Science}

\permanentaddress{Apt \#420, 11121- 82 Ave\\
                  Edmonton, AB \\
                  Canada, T6G 0T4} % each "\\" causes new line,
                                         % no "\\" on the last line.

\examiners{External, Chemical Engineering, University of Waterloo\\\\
			Examiner 2, Computing Science} %Names of people on your committee
                                  %Put external examiners first , followed
                                  %by other committee members.
                                  %Add University if not from University of Alberta
                                  %This is a variable length list

\convocationseason{Fall} % or "Spring", again string appears in thesis.
\convocationyear{\number2013}

% optional quote
\frontpiece{\emph{Machines will be capable, within twenty years, of doing any work that a man can do.}
\begin{flushright}
-- Herbert Simon, 1965.
\end{flushright}
}

% optional dedication
\dedication{\vspace*{20mm}
            \emph{To the Count} \\
            \emph{For teaching me everything I need to know about math.}
}
\begin{document}

\admin % use when you wish to produce the whole works, ie.,
              % approval page, release page ect.

   % environment for abstract.

\begin{abstract}
TODO
\end{abstract}

%\begin{preface} % Optional environent for preface
% ...
%\end{preface}

\doublespacing
\begin{acknowledgements} % environment for acknowledgements.
%\input{ack}
\end{acknowledgements}

\singlespacing
               % Now the table of contents etc.
\tableofcontents
\listoftables  % if you have any
\listoffigures % if you have any
               % minimal support for list of plates and symbols (Optional)
\begin{listofplates}
...            % you are responsible for formatting this page.
\end{listofplates}
\begin{listofsymbols}
...            % You are responsible for formatting this page
\end{listofsymbols}
               % Now a set up command and you are off
\doublespacing % Optional; default is \singlespacing; you can also use
               %                \onehalfspacing or \truedoublespacing
\bodyoftext

%  ... your magnificient thesis ... 
%  hopefully more than two lines! Use standard Latex sectioning commands
%  like \chapter ect. End with the bibliography
\chapter{Introduction}

TODO

\chapter{Background}

This chapter describes the necessary background information for the research.  First in section \ref{sec:data_dependence}, different kinds of data dependences are explained.  The section also gives motivating examples where there are \textit{may dependences} inside loops those change with respect to program inputs.  Section \ref{sec:fdo} gives the background information on feedback directed optimization (FDO).  This section describes different steps of FDO with a relevant discussion on \textit{dependence profiling}.  Section \ref{sec:tls} describes different concepts of \textit{Thread level speculation} (TLS), with  a discussion of special hardware features necessary to support TLS.  The section also describes the features of IBM Bluegene/Q that enables TLS in it.  The TLS specific compiler information and different TLS pragmas are also described.  Lastly, section \ref{sec:llvm} gives informaton about the different useful passes and tools in LLVM that are used in the research.

\section{Data Dependence}
\label{sec:data_dependence}
A data dependency in computer science is a situation in which a program statement (instruction) accesses the same memory location that was accessed by a preceding statement.  If there are two statements S1 and S2, S2 depends on S1 when \\- 

$[R(S1) \bigcap W(S2)] \bigcup  [W(S1) \bigcap R(S2)] \bigcup [W(S1) \bigcap W(S2)] \neq \theta $\\

Where $R(S)$ is the set of memory locations read by a statement $S$, and $W(S)$ is the set of memory locations written by statement $S$.  Dependences exist when there is a feasible runtime execution path from $S1$ to $S2$.  This is called Bernstein Condition, named by A. J. Bernstein. \cite{bernstein}

Dependences are categorized in the following manner - 

\begin{itemize}
\item \textbf{Flow (data) dependence} $W(S1) \bigcap R(S2) \neq \theta $:  $S1$ writes something read by $S2$
\item \textbf{Anti-dependence} $R(S1) \bigcap W(S2) \neq \theta $: $S1$ reads something before $S2$ overwrites it.
\item \textbf{Output dependence} $W(S1) \bigcap W(S2) \neq \theta$: Both $S1$ and $S2$ write to the same memory location.
\end{itemize}

\subsubsection{Dependency in Loops}

Data dependency in loops occur when the same memory location is accessed (write/read) by different statements within the same iteration or by different instances of the same statement or different statements between different iterations of the loop.  Based on the different kinds of dependences that might occur inside loops, the dependences in loops can be classified as the following two types- 

\begin{itemize}
\item \textbf{Loop Independent }: Dependence between statements executed within the same loop iteration
\item \textbf{Loop Carried}: When the same address is accessed by statements or statement instances executed in different iterations.
\end{itemize}

Static dependence analysis is a technique where the compiler determines at compile-time whether there is  any dependence among statements of the program.  Generally the dependence analysis calculation is based on alias analysis.  Two pointers in the program are called \textit{aliases} when they refer to the same variable.  The alias analysis by the compiler can return three types of aliasing relationship - \textit{must}, \textit{may} and \textit{no} alias.  Using the \textit{must} and \textit{no}  aliasing information, the compiler can determine statically whether a loop can be parallelized or not.  But for \textit{may} aliases, the parallelizability of the loop is not statically provable. Figure \ref{fig:example1} gives an example of a program where the alias relationship changes based on the input to the program.  The dependence relationship of the second for loop inside \textit{main} cannot be determined by the compiler and is reported as may dependence. \\


\begin{figure}[h]
\begin{center}
\scalebox{.5}{ \includegraphics{./pdf/pointer.pdf}}
\caption{Code snippet for a loop where the alias relationship can't be determined at compile time.  The second \textit{for} loop in main can be either parallel or not based on the user input \textit{choice}.  Call to the function \textit{change} changes the alias relationship.}
\end{center}
\label{fig:example1}
\end{figure}

%\begin{table}[ht]
%\caption{Motivating Examples}
%\begin{tabular}{c||c}
%
% \includegraphics[scale=0.4]{./pdf/call.pdf} &
%
% \includegraphics[scale=0.4]{./pdf/pointer.pdf} \\
%
%\end{tabular}
%\end{table}
\begin{figure}[h]
\begin{center}
\label{fig:motv_exmp_2}
\scalebox{.5}{ \includegraphics{./pdf/call.pdf}}
\caption{Code snippet for a loop where the alias relationship can't be determined at compile time.  The second \textit{for} loop in main can be either parallel or not based on the user input \textit{choice}.  Here the array subscript is a function call and the dependence analysis techniques reports this as a \textit{may} dependence.}
\end{center}
\end{figure}

Some dependence analysis use algebric techniques to find out the dependence behaviour of the loop. \cite{itest} \cite{lrpdtest} \cite{omega}  These techniques operate of array subscripts.  But they require the array subscripts to be \textit{affine} function of the loop induction variables.  If the subscript accesses another element or is determined by a function call, compilers can not prove the dependence at compile time and they report \textit{may} dependence. Figure \ref{fig:motv_exmp_2} demonstrates a program where the dependence relation of the statement inside the second \textit{for} loop of the \textit{main}  function is not provable at compile time because the subscript of the array is a function call that might change the alias relation(s). The function \textit{change} makes the loop either dependent or parallel based on user input.  Profiling of memory accesses come into play in these situations.

 
\section{Feedback directed Optimization}
\label{sec:fdo}

Feedback directed optimization (FDO) is a compiler optimization technique where the behaviour of the program is observed for some \textit{training} input.  The information gathered in the \textit{profiling} run is kept in a \textit{profile} file.  The profile file is read by the compiler in a subsequent analysis pass where different optimizations can be performed based on the profile information.  A typical feedback directed optimization is done in the following steps -

\begin{itemize}
\item \textbf{Instrumentation}\\

Compiler optimizations work on some Internal Representation (IR) of the source code.  The compiler front end converts the high level constructs of the source code to IR (language independent) for the middle end to work with it.  An instrumentation pass in FDO instruments the bitcode by adding calls to some profiling library functions so that the behaviour of interest can be captured.

\item \textbf{Profiling} \\

The instrumented bitcode is run with some training input and the output is stored in a profile file. 

\item \textbf{Optimization} \\

After the profiling run, an optimization pass is executed that takes both the original bitcode and the profile file as input and applies some code transformation based on the profiling data.  As a result of this pass, an optimized (may not be optimal) version of the bitcode is produced that in turn is translated to an executable by the back end.

\end{itemize} 

\subsection{Dependence Profiling}

There are many behaviours of the program that can be captured with the help of profiling.  Some examples are \textit{edge profiling} (The execution frequency of the edges in the Control Flow Graph(CFG) of the program are profiled), \textit{path profiling} (the execution frequency of the different execution paths in the Call graph of the program are profiled), \textit{execution profile}  (Execution profile of the subroutines in the program that contains the execution time, resource consumption etc. by various routines in the program), \textit{power profiling} (power consumption by programs) etc. \\

\begin{figure}[h]
\begin{center}
\scalebox{.5}{ \includegraphics{./pdf/dep_profiler.pdf}}
\caption{An Example of a dependence profiler}
\end{center}
\label{fig:dep_profiler}
\end{figure}

Dependence profiling stores information about the different memory accesses by program parts.  For loops, the memory address accessed, the type of access (read/ write) and the iteration id for a particular loop is observed to find dependences.  The technique works by putting memory accesses by loads and stores in earlier iterations in a table (typically hash table for fast lookup).  Whenever a new load or store is executed, the table is searched to see whether the same address was accessed in a previous iteration.  If the same memory was accessed before, a new dependence is found. Figure \ref{fig:dep_profiler} demonstrates dependence profiling to find the RAW dependences in a loop.  Calls to the profiling library functions are inserted before every load and store instructions in the bitcode of the loop. For stores, the function logs a tuple <iteration id, memory address> in memory. Whenever a load is executed, the function checks in the log for a matching memory address to find a dependence.

\subsection{Cost of profiling}

Profiling comes with a cost.  There are two types of overhead - 1. space overhead and 2. time overhead.  For dependence profiling, the memory overhead is huge if the memory accesses for the \textit{may dependent} statements across the \textit{whole} iteration space of the loop had to be stored.  Thus to reduce this overhead, the dependence analysis techniques normally work on \textit{loop samples} (some considerable portion of the iteration space) to get an approximation of the dependence behaviour.  Moreover, some compression techniques can also be applied to the storage required for profiling. \\

The timing overhead is optimized by either a fast lookup algorithm (e.g. hash table) and/or by creating smaller search space (with the use of access sets \cite{sd}).  For JIT(Just-in-time) compilers those use runtime profiling, time overhead is a major issue.  But for \textit{off-line} profiling (profiling done in a separate profiling run), the time is not a major issue, though profiling time should be reasonable.

\section{Combined Profiling}

Programs dependence behaviour changes according to the input to the program.  In the programs of figure \ref{fig:example1} and \ref{fig:motv_exmp_2}, for an input of '0', the second \textit{for} loop inside \textit{main}  \textit{cannot} be executed in parallel while for an input of '1', the loop can be parallel. But FDO has not achieved widespread use by compiler users because the selection of a data input to use for profiling that is representative of the execution of the program throughout its lifetime is difficult. For large and complex programs dealing with many use cases and used by a multitude of users, assembling an appropriately representative workload may be a difficult
task. Picking one training run to represent such a space is far more challenging, or potentially impossible, in the presence of mutually-exclusive use cases. Moreover, user workloads are prone to change over time. Performance gains today may not be worth the risk of potentially significant performance degradation in the future.\\

Berube et al \cite{BerubeCP} proposed a method called Combined Profiling(CP) that eases the burden of training-workload selection while also mitigating the potential for performance degradation. First, there is no need to select a single input for training, because data from any number of training runs can be
merged into a combined profile. More importantly, CP preserves variations in execution behaviour between inputs. The distribution of behaviours can be queried and analyzed by the compiler when making code transformation decisions. \\

 The profile of a program records information about a set of program behaviours. A program behaviour
$B$ is a (potentially) dynamic feature of the execution of a program. The observation of a behaviour $B$ at a location $l$ of a representation of the program is denoted $B_l$ A behaviour
$B$ is quantified by some metric $M (B)$ as a tuple of numeric values. A monitor $R(B, l, M )$ is injected into a program at every location $l$ where the behaviour $B$ is to be measured
using metric $M$ . At the completion of a training run, each monitor records the tuple $<l, M (B_l )>$ in a raw profile.  In case of dependence profiling, the \textit{behaviour} to be observed is whether the loop is independent or not.  A \textit{monitor} is inserted inside every loop of the program and the \textit{metric} is the number of independent/ dependent executions.  \\

CP stores the profile information with the help of histograms. Histograms are built in an incremental fashion in CP, thus removing the cost of storing multiple profile files that might increase the storage cost.  In general, updating produces a new histogram in 2 steps:

\begin{enumerate}
\item Determine the range of the combined data. Create a new histogram with this range.
\item Proportionally weight the bins of the new histogram.
\end{enumerate}

\begin{figure}[h]
\begin{center}
\scalebox{.5}{ \includegraphics{./pdf/histogram.pdf}}
\caption{Combining histograms. $H_1$ has a bin width of 10
and a total weight of 12; $H_2$ has a bin width of 8 and a total
weight of 15. The combined histogram $H_3$ has a bin width
of 13 and a total weight of 27.
}
\end{center}
\label{fig:hist}
\end{figure}

The combination of two histograms $H_1$ and $H_2$ into a new histogram $H_3$ is illustrated in Figure \ref{fig:hist}. The range of $H_3$ is simply the minimum encompassing range of the ranges of $H_1$ and $H_2$ : $[min(100, 85), max(150, 125)] = [85, 150]$.
\\
This range is divided into the same number of bins as were present in the original histograms. The weight of a bin $b_i$ of $H3$ is given by the weights of the bins of $H_1$ and $H_2$ that overlap the range of bi multiplied by the overlapping proportion. For example, let $b_3$ be the third bin of H3 in
Figure \ref{fig:hist}. In $H_1$ the bin width is 10, and in $H_2$ the bin width
is 8. The weight of $b_3$ in $H_3$ is calculated as follows-

\begin{center}
$W_{b3}(H1) = (((120-111)\div 10)\times 3) + (((124-120)\div 10)\times 2)  = 3.5 $ \\
\end{center}
\begin{center}
$W_{b3}(H2) = (((117-111)\div 8)\times 1) + (((124-17)\div 8)\times 2)  = 2.5 $ \\
\end{center}
\begin{center}
$W_{b3}(H3) = (3.5 + 2.5) = 6.0$
\end{center}


  
\section{Thread Level Speculation}

This section describes the different concepts of TLS.
\label{sec:tls}

\subsection{Overview}
 
 
\begin{enumerate}
 
\item Predecessor and Successor Threads \\

Under the thread-level speculation (also called speculative parallelization) approach, sequential sections of code are speculatively executed in parallel hoping not to violate any sequential semantics.
The control flow of the sequential code imposes a total order on the threads. At any time during execution, the earliest thread in program order is non-speculative while the later ones can be speculative.
The terms \textit{predecessor} and \textit{successor} are used to relate threads in this total order.  In most schemes a squash rolls the execution back to the start of the thread, but some proposals in the literature use periodic checkpointing of threads such that upon a squash it is only necessary to roll the execution back to the closest safe checkpointed state. When the execution of a nonspeculative thread completes it commits and the values it generated can be moved to safe storage (usually main memory or some shared higher-level cache). At this point its immediate successor acquires non-speculative status and is allowed to commit. When a speculative thread completes it must wait for all predecessors to commit before it can commit.


\item Inter Thread Data Dependence \\

Data dependencies are typically captured by monitoring the data written and the data read by individual threads. A data dependence violation occurs when a thread writes to a location that has already been read by a successor thread. Dependence violations lead to squashing of thread, which involve discarding the side effects produced by the thread being squashed.

\item Buffering of States \\

Stores performed by a speculative thread generate speculative state that cannot be merged with the safe state of the program because this may lead to incorrect results. Such state is stored separately, typically in the cache of the processor. They are not written back to memory. In case of a violation is detected, the state is discarded from the cache. Also if a speculative thread overflows its speculative buffer the thread must stall and wait to become non-speculative. When the thread becomes non-speculative, the state is allowed to propagate (commit) to memory. When a non-speculative thread finishes execution, it commits. Committing informs the rest of the system that the state generated by the task is now part of the safe program state.


\item Data Versioning \\

A thread has at most a single version of any given variable. However, different speculative threads running concurrently in the machine may write to the same variable and, as a result, produce different versions of the variable. Such versions must be buffered separately. Moreover, readers must be provided the correct versions. Finally, as threads commit in order, data versions need to be merged with the safe memory state also in order to ensure correctness.

\item Multi-Versioned Caches \\

 A cache that can hold state from multiple tasks is called multi-versioned. \cite{Cintra00}\\
\cite{Gopal}\cite{steffanISCA00} There are two performance reasons why multi-versioned caches are desirable: they avoid processor stalls when there is imbalance between tasks, and enable lazy commits. 
 
 \end{enumerate}
 
 Speculative threads are usually extracted from either loop iterations or function continuations, without taking into consideration possible data dependence violations. The compiler marks these
structures with a fork-like spawn instruction, so that the execution
of such an instruction leads to a new speculative thread. The parent thread continues execution as normal, while the child thread is mapped to any available core. For loops, spawn points are placed
at the beginning of the loop body, so that each iteration of the loop spawns the next iteration as a speculative thread. Threads formed from iterations of the same loop (and that, thus, have the
same spawn point) are called \textit{sibling threads}. For function calls, spawn points are placed just before the function call such that the non-speculative thread proceeds to the body of the function, and a speculative thread is created from the function’s continuation. \cite{XekalakisICS09} \\

Two different overheads should be considered when running loops speculatively in parallel.  The first overhead comes from buffering the state of the program before the start of speculative execution so that the program can be rolled back to a previous consistent thread.  If the computation done inside a thread is not large enough to overcome this cost, there is no benefit from speculation. \\

The overhead comes from the cost of misspeculation.  If there is actual dependence occurring between threads, the younger thread is squashed.  After a number of retries, the loop becomes sequential.  Thus if the probability is high that the loop is dependent, the loop should not be executed in parallel.
 
\subsection{TLS in IBM BlueGene/Q}

Bluegene/Q(BG/Q) is the latest IBM supercomputer in the Bluegene series (after BG/L and BG/P) that has hardware support for TLS and Transactional Memory(TM).  The initial objective of BG/Q was to include TLS support but later TM support was also added to it.  BG/Q requires hardware support for TLS, working in collaboration with a speculative runtime. The point of coherence for TLS in BG/Q is the L2 cache.  Each different version of a memory address can be stored in a different way of the L2 cache. When a write occurs for a speculative thread, the L2 allocates a new way in the corresponding set for the write. A value stored by a speculative write is private to the thread and is not made visible to other threads. The value is made visible to other threads when a thread commits and is discarded upon a thread squashing. In addition, the L2 directory records, for each memory access, whether it is read or written, and whether it is speculative. For speculative accesses, the hardware also tracks the thread that has read or written the line by recording the speculation ID used by the thread to activate speculation. This enables the hardware to detect conflicts among threads and also between speculative and non-speculative thread. \\

The IBM \textit{xlc} compiler has been modified to give speculation support in BG/Q and it is called \textit{bgxlc\_r}.  The \_r extension at the end generates thread safe code.  The following pragmas are used to speculatively execute a loop in parallel. \\

\textbf{\#pragma speculative for} \\

By default, if this pragma is used, iterations of the loop are divided into \textit{chunks} of size
\textit{ceiling(number\_of\_iterations/number\_of\_threads)}. Each thread is assigned a separate chunk.

The following options can also be used with the \textit{\#pragma speculative for} - \\

$num\_threads (int\_exp)$ - The value of \ is an integer expression that specifies the number of
threads to use for the parallel region.\\

$schedule (type)$- Specifies how iterations of the for loop are divided among available threads.
Thread-level speculative execution supports static scheduling. Acceptable
values for type are as follows:\\

\textit{static} - Iterations of a loop are divided into chunks of size
\textit{ceiling(number\_of\_iterations/number\_of\_threads)}. This is the default scheduling policy. This scheduling policy is also known as block scheduling.\\

\textit{static,n}- Iterations of a loop are divided into chunks of size $n$. Each chunk is
assigned to a thread in round-robin fashion. $n$ must be an integral assignment expression of value 1 or greater. This scheduling policy is also known as block cyclic scheduling. If $n=1$, iterations of a loop are divided into chunks of size 1 and each chunk is assigned to a thread in round-robin fashion. \\

\begin{figure}[h]
\begin{center}
\scalebox{.6}{ \includegraphics{./pdf/indep_window.pdf}}
\caption{An example where every 4 iterations of the loop can be executed in parallel}
\end{center}
\label{fig:indep_window}
\end{figure}


Figure \ref{fig:indep_window} shows an example where there is a \textit{strided} dependence pattern. (Dependence occurring at every fourth iteration of the loop).  Here every 4 iterations can be executed in parallel and into batches. The following pragma should be used in this scenario- \textit{\#pragma speculative for num\_threads(4) schedule(1)}.  While for using 2 threads pragma will be \textit{\#pragma speculative for num\_threads(2) schedule(2)}.  The $num\_threads$ and $schedule$ options should be carefully chosen to balance the overhead of thread creation.




\section{LLVM}
\label{sec:llvm}

The proposed framework has been implemented in the LLVM compiler infrastructure.  LLVM has modular structure that makes the compiler easy to work with.  This section describes the different tools and passes those are useful for loop dependence analysis.

\begin{enumerate}

\item \textbf{The \textit{opt} tool}\\

The \textit{opt} tool of \textit{llvm} is useful to run different optimization-related passes.  There can be \textit{analysis} passes that performs analysis on IR constructs (like loops, CFGs, Call Graphs etc.) but does not modify the IR. The \textit{transformation} passes performs the actual transformations (e.g loop restructuring, Code motion etc.)

\item \textbf{Useful Passes}

The -scev (ScalarEvolution) is an analysis pass that can be used to analyze and categorize scalar expressions in loops.  This pass specializes in recognizing general induction variables, representing them with the abstract and opaque
SCEV class.  Given this analysis, trip counts of loops and other important properties can be obtained. \\

The -mem2reg (Promote Memory to Register) is a transformation pass that converts non-SSA(static single assignment) form of LLVM IR into SSA form, raising loads and stores to stack-allocated values to "registers" (SSA values). Many of LLVM optimization passes operate on the code in SSA form and thus most probably will be no-op seeing IR in non-SSA form.\\

The -loops (Natural Loop Information) is an analysis pass that is used to identify natural loops and determine the loop depth of various nodes of the CFG.\\

The -loop-simplify is a transformation pass that performs several transformations on natural loops to change them into a simpler form, which makes subsequent analyses and transformations simpler and more effective. Loop pre-header insertion guarantees that there is a single, non-critical entry edge from outside of the loop to the loop header.  Loop exit-block insertion guarantees that all exit blocks from the loop (blocks those are outside of the loop that have predecessors inside of the loop) only have predecessors from inside of the loop (and are thus dominated by the loop header). \\

The -memdep pass is an analysis pass that performs memory dependence analysis.  This pass is based on the alias analysis pass.  The following types of alias analysis available in LLVM

\begin{itemize}

\item The -basicaa pass is an aggressive local analysis that knows many important facts like -

\begin{itemize}
\item Distinct globals, stack allocations, and heap allocations can never alias.
\item Globals, stack allocations, and heap allocations never alias the null pointer.
\item Different fields of a structure do not alias.
\item Indexes into arrays with statically differing subscripts cannot alias.
\item Many common standard C library functions never access memory or only read memory.
\item Function calls can not modify or references stack allocations if they never escape from the function that allocates them (a common case for automatic arrays).
\end{itemize}

\item The -globalsmodref-aa pass pass implements a simple context-sensitive mod/ref and alias analysis for internal global variables that don’t “have their address taken”. If a global does not have its address taken, the pass knows that no pointers alias the global. This pass also keeps track of functions that it knows never access memory or never read memory. The real power of this pass is that it provides context-sensitive mod/ref information for call instructions. This allows the optimizer to know that calls to a function do not clobber or read the value of the global, allowing loads and stores to be eliminated.

\item The -steens-aa pass implements a variation on the well-known "Steensgaard’s algorithm" for interprocedural alias analysis. Steensgaard’s algorithm is a unification-based, flow-insensitive, context-insensitive, and field-insensitive alias analysis that is also very scalable (effectively linear time). 

\item The -scev-aa pass implements AliasAnalysis queries by translating them into ScalarEvolution queries. This gives this pass a more complete understanding of pointer instructions and loop induction variables than other alias analyses have.

\end{itemize}

The dependence analysis reports \textit{may}, \textit{must} and \textit{no} dependence information.  Instrumentation is performed for the \textit{may} dependent instructions reported by the dependence analysis.  The next chapter describes the details of how the framework is implemented.

\end{enumerate}

\chapter{Description of the Framework}

\section{Introduction}

The proposed framework has been implemented in the LLVM compiler infrastructure.  A \textit{transformation} instruments the LLVM \textit{Internal Representation}(IR) to prepare the later for profiling. The framework only instruments memory accesses (loads and stores) for loops with some specific properties (described later) as the restrictions imposed by the (\textit{bgxlc\_r}) compiler.  The instrumented code is run for different inputs.  These runs build the combined profile incrementally and writes them in a file called \textit{llvmprof.out} \footnote{The standard name for a LLVM profile file. Note that the profile file will be kept in a spearate directory with a specific program, thus avoiding overwriting}.  Another analysis pass takes this profile file, performs a mapping between the profile information and loops in the program, and produce an \textit{information} file based on the profitability of loops.  The \textit{information} file contains the file names and the line numbers where the speculative \textit{pragma} is to be inserted.  A script takes this information file as input and automatically performs the source-code instrumentation with the TLS \textit{pragmas}. After the instrumentation is done, the optimized source files are compiled with the \textit{bgxlc\_r} compiler to produce the executable which in turn can be run on BG/Q.

In this chapter, section \ref{framework} gives a high-level description of the framework.  Next in section \ref{LLVMImplementation} and \ref{library}, the different components of the framework are explained in details.

\section{The Framework}
\label{framework}

\begin{figure}[ht]
\begin{center}
\scalebox{0.4 }{ \includegraphics{./pdf/framework_org.pdf}}
\renewcommand{\figure}{Fig.}
\caption{ The Framework}
\end{center}
\label{fig:framework}
\end{figure}

Figure \ref{fig:framework} shows the implemented framework.  First a \textit{transformation} pass \textit{profile-dependence} is run on the given source-code to instrument the later for profiling.  Also another \textit{analysis} pass \textit{printDbgInfo} is run to store the loop IDs and their corresponding file names and line numbers in a \textit{log} file.  Both of these passes need the bitcode to be produced with the debug information (-g option of clang).  Also the transformation pass needs three LLVM passes - \textit{mem2reg}, \textit{loops} and \textit{loop-simplify}.  After that, the instrumented bitcode is run with different inputs to produce a combined profile file \textit{llvmprof.out}.  Another analysis pass, \textit{profile-analysis} takes both the profile file and \textit{log} file as input and produces an \textit{information} file about the file name and line numbers where the speculation pragma(s) is/are to be inserted.  A script is then run that automatically instruments the source code following the \textit{information} file. \\

The following sections give more details about the different components of the framework.


\section{The Instrumentation Pass}

The instrumentation pass performs IR instrumentation by adding calls to library-functions used for profiling.  This section describes the implementation details of the Instrumentation Pass.

\subsubsection{Choosing Loops}

The pass is written as a ModulePass in LLVM.  For a given bitcode, the pass iterates through all the modules.  For a given module, the pass iterates over the functions of the module and for a given function, the pass iterates over the basic blocks in the function. \\

In LLVM, the first basic block of a \textit{for} loop is named as \textit{for.cond$X$} where $X$ is any number (depends on the number of loops in the program).  Thus while iterating through the basic blocks of the function, if the pass finds a basic block with \textit{for.cond} in it's name, a variable \textit{loop\_count} is incremented. \\

The \textit{loop\_count} variable is used as a loop-identifier.  The variable is of type unsigned integer.  The variable is also used to create a unique global variable - \textit{iteration\_count} for each loop.  \textit{iteration\_count} is used in identifying the current iteration ID of the loop (as they will be needed later to calculate the dependence distance).  The iteration counter is cleared at the exit block of the loop so that the counter can be reused for another execution of the loop. \\

For being a speculation candidate, the loop has to have the following properties -

\begin{itemize}

\item Branching in or out of structured block and  parallel/work-sharing loop is not allowed. So if the loop has multiple exit blocks, the loop is not a speculation candidate.  Multiple exit blocks exist when there are one or more jumps (e.g. \textit{goto}-s) in the loop body.

\item The loop should be countable.  An example of non-countable loop is given in figure 1.2. 

\begin{figure}[h]
\label{fig:non_countable}
\begin{center}
\scalebox{.6}{ \includegraphics{./pdf/example_uncountable.pdf}}
\renewcommand{\figure}{Fig.}
\caption{ Example of a non-countable loop from \textit{hmmer} benchmark}
\end{center}
\end{figure}

\item If the loop has non-intrinsic function calls inside it's body, then the loop is not safe to parallelize.

\end{itemize}

\subsubsection{Implementation in LLVM}
\label{LLVMImplementation}
Different functions are used to check the above mentioned special characteristics of the loop.  This section gives the details of them.

\begin{itemize}

\item \textbf{Early exit condition of the Loop}\\
LLVM has a function, {\em getUniqueExitBlock}, that returns either the unique exit block of a loop or returns null if the loop has multiple exit blocks.

\item \textbf{Countable Loops}\\
The \textit{getSmallConstantTripCount}() function from the \textit{ScalarEvolution} pass is used to identify countable loops.  The function returns 0 if the trip count is unknown or not constant.

\item \textbf{Function Calls}\\
For checking if the loop body has function calls, the \textit{Call} instructions inside the loop body are identified.  If the call is to a C-library function, the body of the function will not be included in the bit code.  If the \textit{call} instruction is not a C-library call, the body will be there in the bitcode.  The \textit{empty}() function of LLVM's \textit{Function} class is used to make this distinction. If \textit{empty}() returns true, the \textit{call} is to a C-library function otherwise not.\\
After filtering out \textit{calls} to C-library functions, a check is necessary to check whether the function may access memory.  If the function may access memory, the loop is \textit{not} parallelized as the side effects of the function call may alter the dependence behaviour of the loop and an inter-procedural dependence analysis is necessary in that case.


\end{itemize}

\subsubsection{The Algorithm}

The instrumentation pass is implemented as A \textit{ModulePass} in LLVM.  Algorithm \ref{alg:InstrumentAlgo} is used for the instrumentation.  Basically the algorithm iterates over all the modules in the program and next all the functions in the module and lastly all the basic blocks in the function. \\

When the first basic block of the function is met, a void pointer is allocated that is used to store the different memory accesses by the \textit{load} and \textit{store} instructions those \textit{may} be dependent. This pointer is also used as an argument to the profiling functions.[Lines 4-6] \\

If the basic block's name contains "for.cond", the pass detects a loop.  Here we assume that no previous optimization(s) have been applied to the bitcode and all the loops' basic blocks are in the same order as the loops are met in the source code.  If a basic block is met with "for.cond" in its name, the $loopCount$ variable is incremented and two new global variables are also created. [Lines 8-13]

\begin{itemize}

\item \textbf{loop\_id} is an unique loop ID for the \textit{for} loop.
\item \textbf{iteration\_id} is required to identify the iteration number of the memory access.  This variable is incremented as we enter a loop body every time during the loop execution.  Also \textit{iteration\_id} is cleared at the exit from the loop so that the variable can be reused for the next execution of the loop \textit{for} loop.

\end{itemize}

Thus two global variables per loop are used. But only loops that have the properties as described in section \ref{LLVMImplementation} are instrumented.  The function \textit{candidateLoop}() checks for these properties.  The function takes the help of \textit{LoopInfo} and \textit{ScalarEvolution} passes of LLVM. \\

After creation of the global variables, the exit block of the loop is identified using the \textit{getUniqueExitBlock}() function.  As the loop is previously checked for having one exit block only, this function always returns a unique basic block.  Next, two instructions are inserted in the basic block.  The first one clears the \textit{iteration\_id} so that the variable can be reused in the next execution.  Another \textit{call} instruction  to the \textit{analyseAndWrite}() function is inserted. This function analyses the stored accesses to find the dependence pattern and write to profile  information. [Lines 14-17] \\

Next the loads and stores are identified and calls to the profiling functions are inserted accordingly.  There can be two types of basic blocks where the \textit{loads} and \textit{stores} can reside.  Either \textit{for.body} that comes just after the loop-conditional checking and \textit{for.end} that is the last basic block of the loop. \footnote{This condition happens when there is loop nesting and there are instructions in the outer loop after the execution of the inner loop}.  The loads and stores in these basic blocks are checked to see whether they are reported as \textit{may} dependent by the static analysis.  If they are, their memory accesses are stored in the \textit{void} pointer created before and a \textit{call} instruction to the profiling functions are inserted.  This function takes the memory address accessed, loop ID, iteration ID and the type of access (load/store) as parameters. More on this function is described later. [Lines 19 -29] \\

If the basic block is the loop body (\textit{for.body}), instructions to increment the \textit{iteration\_id} for the specific loop are also inserted. [Lines 30-33] \\

The collected profile is written to the profile file only once - during the exit from the program.  Now there can be two ways how a program can exit - 

\begin{itemize}

\item Return from \textit{main}
\item Calling the \textit{exit}() function from any functions.

\end{itemize}

So the instructions are checked in the basic block to see whether that is a \textit{return} instruction from \textit{main} or they are \textit{call} instructions to \textit{exit}() function.  If such instructions are met, a call to the function that accesses the profile file and write the collected profile information from memory to file is called. [Lines 35 - 49] \\
\begin{algorithm}
\begin{algorithmic}[1]
\FORALL {$mod$ in $Modules$}
	\FORALL {$func$ in $mod$}
		\FORALL {$basic\_block$ in $func$}
			\IF {$basic\_block$.$isFirstBasicBlockOfFunction()$}
				\STATE $allocateVoidPointer()$
			\ENDIF 
			\IF{$basic\_block.getName().contains(for.cond)$}
				\STATE loopCount++
				\STATE $loop$ = $getLoopFor$($basic\_block$)
				\IF {$loop$}
					\IF {$candidateLoop(loop)$}
						\STATE $createGlobal( loop\_id )$
						\STATE $createGlobal( iteration\_id )$
						\STATE $endBB$ = $getUniqueExitBlock(loop)$
						\STATE $createInstruction(makeIterationCountZero)$
						\STATE $createInstruction(callAnalyseAndWrite)$
						\STATE $insertInstructions()$
						\STATE
						\IF {$basic\_block.isLoopBody() || basic\_block.isLoopEnd()$}
							\FORALL {$instruction$ in $basic\_block$}
								\IF{$isMayLoad() || isMayStore()$}
									\STATE $createInstruction(getAddress(void_p))$
									\STATE $createInstruction(getLoopID())$
									\STATE $createInstruction(getIterationID())$
									\STATE $createInstruction(profileAccess())$
									\STATE $insertInstructions()$
								\ENDIF
							\ENDFOR
						\ENDIF
						\IF {$basic\_block.isLoopBody()$}
							\STATE {$createInstruction(incrementIterationCount)$}
							\STATE {$insertInstructions()$}
						\ENDIF
					\ENDIF
					
					\IF {$func == main$}
						\FORALL {$inst$ in $basic\_block$}
							\IF {$inst.isReturn() || inst.isCallToExit()$}
								\STATE $createInstruction(callWriteToFile())$
								\STATE $insertInstructions()$
							\ENDIF
						\ENDFOR	
					\ELSE				
						\FORALL {$inst$ in $basic\_block$}
							\IF {$inst.isCallToExit()$}
								\STATE $createInstruction(callWriteToFile())$
								\STATE $insertInstructions()$
							\ENDIF
						\ENDFOR
					\ENDIF

				\ENDIF
			\ENDIF
		\ENDFOR
		
	\ENDFOR
\ENDFOR
\caption{Instrumentation Algorithm}
\label{alg:InstrumentAlgo}
\end{algorithmic}
\end{algorithm}



\section{The Profiling Library}
\label{library}

In this section, the different functions and the data structures used in the profiling library are described.  

\subsubsection{Functions}

For each load and store instruction those are reported as may dependent by the static dependence analysis, a function call is inserted that behaves differently for loads and stores.  The following functions are the most important functions of the library. \\   


\textit{void common(void * x,int count,int loop, char type)} is the main function that is used for dependence detection.  The working algorithm for this function is given in Algorithm \ref{alg:common_algo}. \\

The dependence behaviour is estimated from a portion of the loop's iteration space, the \textit{loop sample}.  The number of iterations to be considered for estimation is tunable by the macro $MAX\_ITERATIONS$.  When this function is called for the first time, it performs a one time initialization of the data structures, this memory portion remains fixed for the whole profiling run.  This is static memory because dynamic memory location gives lots of memory fragmentation. \\

Until the \textit{loop sample} size is met, when a store in encountered, it is thrown in a bag, by the function $add\_store$.  Since we only care about loads that depend on previous stores (RAW dependence), whenever a load is encountered, it is checked against the already captured store instructions for a given loop.  If the access addresses are same, a \textit{dependence pair} is found.  A dependence pair is defined as a tuple $(i_1,i_2)$ for a given loop where $i_1$ is the iteration ID of the store and $i_2$ is the iteration ID of the load. Note, $i_1$ is always less than $i_2$. Dependence pairs for that execution for that given loop are also kept.\\

When the loop enters an iteration ID that just exceeds the sample size; the collection of dependence pairs are analyzed to find the dependence behaviour.  If there are no dependence pairs, that means there were no RAW dependences for that execution of the loop and the loop's parallel execution count is incremented.  If there are some dependence pairs, the loop was not parallel for that execution.  As seen in figure \ref{fig:indep_window}, strided memory access patterns are interesting cases for parallel execution. The function $check\_stride()$ checks if there was a strided dependence pattern. The function uses the state machine described in figure \ref{fig:stride} to find strided dependence.  First the function calculates a dependence distance (distance between the iteration id of the write and that of read) for a given dependence pair.  Next the function checks for dependence pairs with the same dependence distance.  Every time such pair is encountered, a counter is increased.  If the counter value reaches a certain threshold (the threshold id set as 5) is reached, a strided dependence is reported, else the dependence becomes irregular. If there was a stride, the value of the stride is stored in the profile information, else the the loop has irregular dependence pattern. Also to reduce the search space, already visited dependence pairs are not searched again.\\

\begin{figure}[h]
\begin{center}
\scalebox{0.5}{ \includegraphics{./pdf/stride.pdf}}
\renewcommand{\figure}{Fig.}
\caption{ The State Machine for detecting strides}
\end{center}
\label{fig:stride}
\end{figure}



After analyzing and storing the dependence pattern in the memory for the given loop, all the data structures used in this execution of the loop is cleared so that they can be reused in the next run.  This is achieved by just clearing some counters, but not resetting the whole data structures.  Also the \textit{flag}, indicating that the dependence analysis is done, is set.

\textit{int analyse\_and\_write(int loop)} is used for the loops whose iteration count is smaller than \textit{MAX\_ITERATIONS}.  This function is inserted at the exit block of the loop.  If the \textit{flag} indicating the dependence analysis is not set, then the function does the same operations as \textit{common()} does for iteration ids less than \textit{MAX\_ITERATIONS}.  Else the function just resets that flag for the next loop execution.  This function is useful for calculating dependence information for loops smaller than \textit{MAX\_ITERATIONS}.

\textit{int write\_to\_file()}  is called only when the program exists.  If there is already a profile file, this function updates the file using the information from the current run.  Else, the function creates a new profile file \textit{llvmprof.out}.  Call to this function is only inserted before the \textit{return} instruction from \textit{main} and before the \textit{call} instruction to the \textit{exit} function.  This function helps in reducing the cost of reading the profile each time a loop terminates.

\begin{algorithm}
\begin{algorithmic}[1]
\caption{Algorithm for the $common$ function}
\label{alg:common_algo}
\IF{$iteration\_id <= MAX\_ITERATIONS$}
	\IF{data structures uninitialized}
		\STATE $initializeDS()$
		\STATE $setInitializedFlag()$
	\ENDIF	
	\IF {$load$ instruction}
		\STATE $check\_dependence()$
	\ELSE
		\STATE $add\_store()$
	\ENDIF
\ELSE
	\IF{$iteration\_id <= MAX\_ITERATIONS+1$}
		\IF {$dep\_pair\_count == 0$}
			\STATE loop parallel
		\ELSE
			\STATE $check\_stride()$
			\IF{stride found}
				\STATE strided dependence
			\ELSE
				\STATE irregular dependence
			\ENDIF
		\ENDIF
		\STATE $set\_analyzed\_flag()$
		\STATE $reset\_DS()$
	\ENDIF
\ENDIF

\end{algorithmic}
\end{algorithm}



\subsubsection{Data Structures}

This sections describes the various data structures those are used by the functions in the profiling library.\\
The data structure used by the library are \textit{static} arrays of structures.  Linked list is not a good choice here due to 1) The fragmentation created in the memory 2) They need an extra storage for storing the pointers to next and previous elements.  The different structures used are as follows. 


\begin{itemize}
\item \textbf{Memory Accesses} \\

struct Access \\
\{\\
	\hspace*{1 cm} int iteration\_id; \\
	\hspace*{1 cm} void * address;\\
\} \\
Only \textit{stores} need to be stored as the loads are checked on fly.  The $iteration\_id$ variable will always be less than $MAX\_ITERATIONS$.  A \textit{void} pointer is used to store memory address of any type.  A static array \textit{struct Access  stores [MAX\_LOOPS][MAX\_ACCESSES] }is used to store this information.  The $MAX\_LOOPS$ parameter is tunable and it gives the upper bound of the number of loops in a program. The value is set to 5000. $MAX\_ACCESSES$ gives the upper bound of the number of stores that can occur for an execution of the \textit{loop sample}.  The value is set to 2000.

\item \textbf{Dependence Pair} \\

struct DependencePair\\
\{\\
	\hspace*{1 cm} int write;\\
	\hspace*{1 cm} int read;\\
	\hspace*{1 cm} char checked;\\
\};\\

\textit{DependencePair} structure has three members.  $write$ and $read$ are used to store the iteration id of the store and load respectively.  $checked$ is a flag that is used during the stride calculation to avoid redundant computation.  A similar array \textit{struct DependencePair  pairs [MAX\_LOOPS][MAX\_DEPENDENCES]} is used to store the dependence pairs. 
\\
\textit{MAX\_DEPENDENCES} is also set to 2000.

\item \textbf{Profile Information} \\

struct Loop\_Info\\
\{\\
	\hspace*{1 cm} int loop\_id;\\
	\hspace*{1 cm} short parallel, irregular, first\_bin, second\_bin, third\_bin, fourth\_bin, fifth\_bin;\\
	
\};\\

$loop\_id$ stores the unique ID of the loop. $parallel$ and $irregular$ variables are used to store the number of parallel executions and the number of executions with irregular dependence.  The bins of the histogram are used to store the stride values for dependence.  An array \textit{struct Loop\_Info  info [MAX\_LOOPS]} is used to store the profile information in the memory.

\item \textbf{Miscellaneous Data Structures} \\

Apart from the main data structure mentioned above, the following arrays are also used -

\begin{enumerate}
\item To keep of the already analyzed \textit{flag}.
\item To keep track of the number of memory access count and dependence pair count.  These counters reduce the costly traversal of the arrays of data structures to prepare them for the next execution of the loop.
\item To store the discovered stride value, if applicable.
\end{enumerate}
\end{itemize}

The total size of the data structures remains fixed as they are initialized once and reused throughout the execution of the program.  For 5000 loops, 2000 accesses and 2000 dependence pairs, the memory overhead comes to 200 MB.

\section{The Profile file}

The profile file produced by the framework stores the dependence information for a number of executions for each loop. For each loop, the file stores the unique loop ID and the number of independent (parallel) and irregularly dependent executions.  The file also stores the discovered stride values in a histogram with five bins. 
Figure \ref{fig:profile_file} shows a sample profile file.\\

\begin{figure}[h]
\begin{center}
\scalebox{0.5}{ \includegraphics{./pdf/profile_file.pdf}}
\renewcommand{\figure}{Fig.}
\caption{ The structure of the profile file.  There are eight columns which represent loop\_id  parallel\_executions  irregular\_executions  first\_bin  second\_bin  third\_bin  fourth\_bin  fifth\_bin. The bins construct the histogram for storing stride values}
\end{center}
\label{fig:profile_file}
\end{figure}

\chapter{Experimental Evaluation}

This chapter describes the experimental evaluation of benchmarks using the LLVM compiler in the BlueGene/Q machine.  The initial experiments are done using the Polyhedral Dependence analyzer in LLVM, called \textit{Polly}.  First, using a simple heuristics, loops with \textit{may dependences} are allowed to speculatively execute in parallel with a hope to get performance improvement.  The heuristic is then modified to filter out cold loops and, with the help of profiling, loops where may dependences materialize at runtime.  The initial experimentation takes the sequential version of the program at the lowest optimization level and does not consider the effect of SIMDization or other loop transformation performed at higher levels of optimization.  A performance comparison with the automatic OpenMP parallelized code by Polly is also done.\\

In the second set of experiments, the dependence analysis pass of LLVM was used and also the effect of higher optimization levels of the compiler and the effect of automatic SIMDization is studied.  A comparison among three parallel version of the code - SIMDized, automatic OpenMP parallelized and speculatively parallelized is also done. The benchmarks chosen were the ones used in TLS literature before.

\section{Variability of Loops' Dependence Behaviour based on Inputs}

One important finding of this research is that the dependence behaviour of loops that has the properties to be speculation candidates, \textit{doesn't change} with respect to the different inputs to the program.  The dependence behaviour of loops from the benchmarks that were used in TLS research before (and are available) is observed for different inputs and \textit{there was not a single loop that has the specific properties for being a TLS candidate and whose dependence behaviour varies with different inputs}.  This finding allows to use a simple heuristic for selection of speculation candidates based on profiling.  If the loop is dependent, the loop is not speculated.  But if the loops is found to be independent, it is speculatively parallelized.  Still more speculation candidates are discovered using this heuristic because all of these candidate loops have \textit{may dependences} which prevents the compiler from parallelizing them.  Table~\ref{table:benchmarks_observed} gives the list of benchmarks that are investigated to find loops with varied dependence behaviour.\\

\begin{table}
\centering
\caption{The Benchmarks observed to find loops with varied dependence behaviour }
\begin{tabular}{|c|c||c|c|} \hline
Benchmark Suite & Benchmark Name & Benchmark Suite & Benchmark Name \\ \hline 
Spec2006 & lbm & Spec2006 & h264ref \\ \hline
Spec2006 & hmmer & Spec2006 & mcf \\ \hline
Spec2006 & sjeng & Spec2006 & sphinx3  \\ \hline
Spec2006 & bzip2   & Spec2006 & gobmk  \\ \hline
Spec2006 & milc  & Spec2006 & namd   \\ \hline
PolyBench/C & 2mm	 &PolyBench/C & 3mm \\ \hline
PolyBench/C & gemm& PolyBench/C & gramschmidt\\ \hline
PolyBench/C & jacobi &PolyBench/C & lu	 \\ \hline
PolyBench/C & seidel &PolyBench/C & cholesky \\ \hline
PolyBench/C & dynprog &PolyBench/C & fdtd\_2d  \\ \hline
BioBench & mummer  &BioBench & protdist  \\ \hline
BioBench & protpar  &BioBench & dnapars  \\ \hline
BioBench & dnamove  &BioBench & dnapenny  \\ \hline
BioBench & dnacomp  &BioBench & dnainvar  \\ \hline
BioBench & dnaml  &BioBench & dnaml2  \\ \hline
BioBench & dnamlk  &BioBench & dnamlk2  \\ \hline
BioBench & dnadist  &BioBench & dollop  \\ \hline
BioBench & dolmove  &BioBench & dolpenny  \\ \hline
BioBench & restml  &BioBench & restml2  \\ \hline
BioBench & seqboot  &BioBench & fitch  \\ \hline
BioBench & kitsch  &BioBench & neighbor  \\ \hline
BioBench & gendist &BioBench & tigr  \\ \hline
BioBench & clustalw &BioBench & hmmer  \\ \hline
NAS & BT & NAS & CG  \\ \hline
NAS & DC & NAS & EP  \\ \hline
NAS & FT & NAS & IS  \\ \hline
NAS & LU & NAS & MG  \\ \hline
NAS & SP & NAS & UA  \\ \hline
\hline\end{tabular}
\label{table:benchmarks_observed}
\end{table}

\section{Speculative Parallelization using the polyhedral dependence analyzer of LLVM}
\label{section:heuristics}

For the initial experimentation, the following two heuristics are used.  

\subsection{Heuristic 1}

According to the first heuristic,  for being a speculation candidate a \textit{SCoP} (loop) should have only \textit{may dependences}. The goal of this heuristic is to relax the constraint for OpenMP parallelization (OpenMP does not parallelize loops with \textit{may dependences}) and find more parallelization candidates.  The hope is that the \textit{may dependences} will not materialize at run time, thus resulting in speedup. 
 
\subsection{Heuristic 2}

Heuristic 1 allows loops with \textit{may dependences} to execute in parallel.  There can be two cases where the overhead of speculation can still negate the gain from parallelism. In heuristic 2, the two criteria for filtering are based on two different overheads. The first criteria considers the overhead from mispeculation and recovery while the second criteria considers the overhead from thread creation and storing the program state so that the system can be rolled back to a consistent state in case of mispeculation.  In this way loops that can not be benefited from SE are filtered. 

\begin{enumerate}
\item In the loops where \textit{Polly} reports only \textit{may dependences}, the memory accesses are profiled for a training run for some inputs.  If the training run shows that the \textit{may dependences} are materializing into \textit{true dependence}(Here true dependence means dependence occurring at run time, not \textit{read after write} (RAW) dependence) at run time, the loop is not parallelized.
\item If the execution time of the loop is less than some threshold as compared to the total execution time of the program, the loop also is not speculatively parallelized.  As in this case, the overhead from speculation can neutralize or worse, negate the gain from parallelism. This threshold is set as 20\% of the total execution time because experimental evaluation shows that allowing  the speculative execution of smaller loops for the benchmarks leads to slowdown. 
\end{enumerate}

\section{The Framework for Polly}
\label{section:framework}

The framework is implemented in the LLVM ~\cite{llvm} compiler infrastructure. Polyhedral dependence analysis is already implemented as a part of the LLVM project and it is called \textit{Polly}.~\cite{grosserImpact11} \\

First the source-code file to be optimized is compiled with LLVM keeping the debug information in the bitcode (with -g option). The debug information is necessary for source-code instrumentation.  Two LLVM passes are run on the generated bitcode. The \textit{printBlocks} pass prints out all the basic blocks in the source file and their corresponding file name and line numbers.\\

Polly's dependence analysis pass \textit{Polly}-\textit{dependences} is also run on the bitcode.  This pass provides information about the SCoPs and the \textit{may} and \textit{must dependences} in them.  This dependence information is extracted from the output and kept into temporary storage.  An analyzer program takes the information generated by the above two passes as input, performs a three-way mapping (SCoPs with basic block information, basic block with file name and line numbers information and SCoPs and dependence information), performs analysis based on which heuristic to use and spits out information about the file names and line numbers where the speculative \textit{pragmas} to be inserted. A script is run that reads the file name and line numbers and instruments the code accordingly.  For the benchmarks tested, the number of loops never exceeds 6000 (the maximum number of candidate loops for speculation are much less than that) and the temporary storage necessary was less than 5 MB.\\

The instrumented source code is then compiled with the \textit{bgxlc\_r} (\_r option generates thread-safe code. \textit{bgxlc} is the IBM xlc compiler specific for the BlueGene/Q machine) compiler to generate the executable that can be run on the BlueGene/Q ~\cite{BGQ} machine.  The framework is fully automatic.

\section{Experimentation}

Experimentation is performed on two different set of benchmarks - SPEC2006 benchmarks~\cite{spec}  and the PolyBench/C benchmarks~\cite{polybench}.  SPEC2006 is chosen because it has been used by other researchers for the evaluation of speculative execution. All the SPEC2006 benchmarks are not reported because some of them don't run successfully on BlueGene/Q.  PolyBench/C benchmarks are chosen because they are suitable benchmarks for polyhedral analysis.  Table \ref{table:bgq_config} shows the hardware details of a BlueGene/Q chip. \\
 
The SPEC2006 benchmarks are run with the \textit{ref, train and test} input and the PolyBench/C benchmarks are run using varied problem size.  For calculating the speedups, each benchmark is run 10 times for a given input and the average running time from the 10 runs were taken. Also the 95\% confidence interval is shown in the bar chart for the 10 runs.\\

Instrumentation in the source code is done automatically following the information file generated by the framework.  The \textit{\#pragma speculative for} pragma from the IBM \textit{bgxlc\_r} compiler was used for speculative execution of loops.  This pragma divides the iterations of a loop into chunks of size
$ceiling(number\_of\_iterations/number\_of\_threads)$. \\

The loop must be countable at compile time to be a speculation candidate.  Each thread is assigned a separate chunk.\\

The baseline for comparison is the sequential version of the benchmarks compiled with the -O0 optimization level of the \textit{bgxlc\_r} compiler.  The lowest optimization level ensures that the optimization of hot loops (-qhot) and automatic SIMDization of the sequential code (-qsimd=auto) is turned off. For comparison with the OpenMP version, the automatic OpenMP code generated by \textit{Polly} for the benchmarks is used.  Polly inserts calls to OpenMP runtime functions to automatically parallelize independent SCoPs. Still the automatic OpenMP version generated by Polly can be worse than the OpenMP version of the code written by the programmer.


\begin{table*}
\centering
\caption{Configuration of a BluGene/Q chip}
\begin{tabular}{|c||c|} \hline
\#Processors&17(16 User and 1 service PowerPC)\\ \hline 
Multithreading&4-way Multithreaded \\ \hline
Clock&1.6GHz \\ \hline
L1 I/D Cache&16KB/16KB \\ \hline
Peak Performance & 204.8 GFLOPS \@ 55W \\ \hline
RAM&16 GB DDR3 \\ \hline
Multiversioned Cache & Support for Transactional Memory and Speculative Execution \\ \hline
L2 Cache & Centrally shared, 32 MB \\ \hline
Chip-to-chip networking & 5D Torus topology + external link \\ \hline
\hline\end{tabular}
\label{table:bgq_config}
\end{table*}

\section{Results}
\label{section:results}

This section describes the experimental results for the initial experimentation.  First the effect of heuristic 1 on the SPEC2006 and the PolyBench/C benchmarks is described.  Then a comparison is made between OpenMP parallelization and the parallelization done by the framework following heuristic 1.  After that the effect of heuristic 2 is shown.  The scalability of the TLS parallelization is also shown.

\subsection{Heuristic 1}

\subsubsection{Spec2006 Benchmarks}

\begin{figure*}
\begin{center}
\includegraphics[scale =0.75]{./pdf/speedup_spec2006.pdf}
\caption{Speed up of the instrumented SPEC2006 benchmarks over the optimized sequential version on a 8 node cluster. Most of them gets a speedup, the maximum being \textit{lbm} due to the presence of a large parallelizable loop. \textit{gobmk} suffers a slow down because of the presence of many loops with small iteration count. \textit{sjeng} also offers a slow down because the \textit{may} dependences actually occur during run time. Dependence profiling can be done to eliminate those candidate loops with \textit{may dependences} for \textit{sjeng}.}
\end{center}
\label{fig:speedup_spec2006}
\end{figure*}

In Figure~\ref{fig:speedup_spec2006}, most of the SPEC2006 benchmarks achieve a speedup over the optimized sequential version.  \textit{lbm} contains loops with no inter-thread data dependences, but these dependences are not statically provable by the compiler and that's why they are not parallelized by OpenMP. This benchmark can be greatly benefited by the speculative execution and obtains the highest speedup because these dependence don't materialize at run time.\\

\textit{gobmk} has many loops with small iteration counts. Small loops are not good candidates for speculative execution because the thread creation overhead negates the impact of parallel execution and we get a slowdown. These loops are later filtered out by heuristic 2.\\

\textit{sjeng} contains loops those are reported as speculation candidates by heuristic 1 because of the occurrence of \textit{may dependences}.  But these dependences actually occur at run time and the overhead from mispeculation prevents this benchmark from achieving speedup.  These loops are not suited for speculative execution and they are eliminated by heuristic 2.\\

Overall, heuristic 1 performs well for most of the SPEC2006 benchmarks.  A 4x speedup is achieved for \textit{lbm} benchmark on a 8-node cluster. 

\begin{figure}[h]
\begin{center}
\includegraphics[scale =0.75]{./pdf/speedup_polybench.pdf}
\caption{Speed up of the instrumented PolyBench/C benchmarks over the optimized sequential version on a 12 node cluster. The last five benchmarks suffers a slow down because of launcing threads for loops with small iteration count and also parallelizing loops in the innermost level (finer granularity)}
\end{center}
\label{fig:speedup_polybench}
\end{figure}

\subsubsection{PolyBench/C Benchmarks}

The speedups reported in Figure~\ref{fig:speedup_polybench} indicate that the  PolyBench/C programs can be divided into two classes according to the effectiveness of thread-level speculation. Class 1 contains programs\textit{ 2mm, 3mm, correlation, covariance, doitgen, gemm} that achieves speed up on a speculative execution and Class 2, containing \textit{gramschmidt, jacobi-2d-imper, lu, ludcmp, seidel} experiences a slow down.  In the Class 2 PolyBench/C programs the loops parallelized are very small and constitutes a very small portion of the overall program execution (mainly initialization arrays).\footnote{These results confirm the finding of Kim et al.~\cite{KimCGO12}}  Therefore the overhead for thread creation in the speculative execution negates the performance achieved from the parallel execution of these loops. Table \ref{table:poly_coverage} shows the percentage of the whole program execution time the parallelized loops take for the PolyBench/C benchmarks. For \textit{seidel} the coverage is only 0.04\% and therefore the loops are not good speculation candidates for this benchmark.  The cold loops are eliminated by heuristic 2.


\subsection{Comparison with OpenMP Parallelization}

Heuristic 1 relaxes the constraint of OpenMP parallelization and allows loops with \textit{may dependences} and no \textit{must dependences} to speculatively run in parallel too.  As seen in Table ~\ref{table:openmp_vs_spec_poly}, heuristic 1 is able to find more parallelization candidates than OpenMP parallelization for the PolyBench/C benchmarks.  Figure~\ref{fig:openmp_vs_spec_poly} shows the speedup gained from OpenMP parallelization and speculative parallelization for the PolyBench/C benchmarks.  For \textit{2mm} and \textit{3mm}, speculative optimization does not give better speedup because no new loops are discovered by the framework.  For the last five benchmarks, OpenMP performs better than speculative optimization because the loops parallelized by heuristic 1 in the benchmarks are pretty small and they take very small portion of the benchmark execution time.  Therefore executing them speculatively causes the overhead from TLS to negate the gain from parallel execution.  The slowdown in these benchmarks are the motivation for heuristic 2.\\

Most of the time the framework outperforms OpenMP for the SPEC2006 benchmarks, as can be seen in Figure~\ref{fig:openmp_vs_spec}.

\begin{table}
\centering
\caption{Number of Loops parallelized by OpenMP parallelism vs Speculative parallelism using heuristic 1.  As heuristic 1 allows loops with \textit{may dependences} and no \textit{must dependences} to be executed in parallel as well, the heuristic was able to find more parallelizable loops.}
\begin{tabular}{|c||c|c|c|c|} \hline
Benchmark &Total & OpenMP & Speculative & Coverage\\ \hline 
lbm & 23 & 1	 & 4 & 97 \\ \hline
h264ref & 1870 & 250	 & 120 & 79  \\ \hline
hmmer &	851	 & 30 & 45 & 72 \\ \hline
mcf & 52 & 2 & 4 & 60 \\ \hline
sjeng &	254 & 16	 & 3 & 12 \\ \hline
sphinx3 & 609 & 11 & 2 & 91 \\ \hline
bzip2 & 232 & 4 & 2 & 35 \\ \hline
gobmk & 1265 & 0 & 1 & 13 \\ \hline
milc & 421 & 5 & 20 & 68 \\ \hline
namd & 619 & 7& 20 &92 \\ \hline
2mm	&20& 7 &7 & 99 \\ \hline
3mm	&27& 10&10 & 99 \\ \hline
gemm	 &13& 3&4 & 98 \\ \hline
gramschmidt	&10& 2&3 & 1 \\ \hline
jacobi & 9&	3& 3 & 2 \\ \hline
lu	& 8 & 1 & 2 & 2 \\ \hline
seidel&	7& 1 & 1 & 0.04 \\ \hline
cholesky & 9 & 0 & 1 & 2  \\ \hline
dynprog & 9 & 7 & 2 & 13 \\ \hline
fdtd\_2d & 14 & 2 & 0 & - \\ \hline
\hline\end{tabular}
\label{table:coverage_1}
\end{table}

\begin{figure*}
\centering
\includegraphics[scale=0.75]{./pdf/openmp_vs_spec}
\caption{Comparison of Speedups gained from OpenMP parallelism done by \textit{Polly} and speculative parallelism for SPEC2006 benchmarks. Speculative parallelization is able to detect more parallelizable loops than OpenMP and gets better speedup. The best improvement is gained for \textit{lbm} as this benchmark has a large loop that can be speculative parallelized but is not parallelized by OpenMP}
\label{fig:openmp_vs_spec}
\end{figure*}

\begin{figure*}[h]
\centering
\includegraphics[scale = 0.75]{./pdf/openmp_vs_spec_poly}
\caption{Comparison of Speedups gained from OpenMP parallelism done by \textit{Polly} and speculative parallelism for PolyBench/C benchmarks in a 12-node cluster. The last five benchmarks experiences a slowdown as compared to OpenMP parallelization because the loops parallelized are small and takes very small portion of the total program execution time.  Therefore the speculation overhead negates the performance gained from parallelism.}
\label{fig:openmp_vs_spec_poly}
\end{figure*}

\subsection{Heuristic 2}

\subsubsection{Profiling and Filtering hot loops}

To tackle the slowdowns, the reported \textit{may dependences} are profiled to see whether the dependences occur at run time. Also the loops that does not take significant amount of the program execution time were not speculatively parallelized because they negate the gain of parallelization due to thread creation overhead. \\

The results shown in Figure~\ref{fig:heu2} indicate that the benchmarks that were experiencing slowdown as compared to OpenMP performs equal or better  when heuristic 2 is used. \textit{Gramschmidt} and \textit{sjeng} performs better than OpenMP as the overhead from parallelizing small loops goes away.  This indicates that by extending heuristic 1, equal or better performance can be achieved as compared to OpenMP.

\begin{figure}
\centering
\includegraphics[scale=0.75]{./pdf/heu2.pdf}
\caption{Comparison of Speedups gained from heuristic 2 vs OpenMP for the slowdown benchmarks. Heuristic 2 performs equal or better than OpenMP for \textit{all} cases after profiling dependences and preventing cold loops and loops with a run time dependence. }
\label{fig:heu2}
\end{figure}

\section{Going Further: Using LLVM's New Dependence Analysis and Comparison with AutoSIMD and AutoOpenMP by the xlc compiler}

For the next set of experiments, the effect of higher optimization levels of the \textit{bgxlc\_r} compiler is observed.  The optimization level used in both the sequential and parallel version is -O5.  By default -O5 level turns on the automatic vectorization (SIMDization) and the various optimizations of hot loops (e.g. loop unrolling etc.).  \\

Also as LLVM's dependence analysis pass has been improved by the time these experiments are performed, the result from the dependence analysis pass is used for profiling.  The SPEC2006 and PolyBench/C benchmarks are used for experimentation.

The baseline for comparison for these set of experiments is the sequential version of the code optimized at the highest level (-O5) with the automatic vectorization and optimization of hot loops turned off.  Comparison is made with three parallel versions of the code.  The automatic SIMDized code, the automatic OpenMP parallelized code and the speculative parallel code generated by the framework.  The following compiler options are used to generate the sequential and the three parallel versions of the code. \\

\begin{itemize}
\item \textbf{Optimized Sequential version:} bgxlc -O5 -qsimd=noauto -qnohot -qstrict 
\item \textbf{Automatic SIMDized version:} bgxlc -O5 -qsimd=auto -qhot -qstrict 
\item \textbf{Automatic OpenMP parallelized version:} bgxlc\_r -O5 -qsimd=auto -qhot -qsmp=auto -qstrict 
\item \textbf{Speculative parallelized version:} bgxlc\_r -O5 -qsimd=auto -qhot -qsmp=auto:speculative -qstrict
\end{itemize}

The \-qstrict compiler option is used to maintain the correct semantics of the program after higher level optimizations.  Because, the higher level optimizations may alter the semantics of the program. \\

\subsubsection{SPEC2006}

\begin{figure}[h]
\centering
\includegraphics[scale=0.56]{./pdf/spec2006_O5.pdf}
\caption{Comparison of Speedups gained from different parallelization techniques as compared to the sequential version of the code optimized at the highest level for SPEC2006 Benchmarks }
\label{fig:speedup_O5}
\end{figure}

Figure~\ref{fig:speedup_O5} gives the speedup obtained from different parallelization of the SPEC2006 benchmarks with the optimized sequential version as baseline. The performance improvement of speculative parallelization over the sequential version optimized at the highest level (-O5) is much less than what was achieved over the sequential version with no optimization (-O0). \\

As seen in the figure, \textit{milc} and \textit{namd} shows similar trends with the three types of parallelization.  The SIMDized version performs better than the optimized sequential version, OMP and speculative execution keep performing better as they are applied to the code.  For \textit{hmmer} and \textit{mcf}, the SIMDization does not improve the performance but still these two benchmarks get some performance improvement while OMP and speculative parallelization is applied. \textit{sphinx3} does not get any improvement for SIMDization, but OpenMP helps here while speculative parallelization does not.  For \textit{bzip2}, OpenMP does not help over SIMD, but speculative parallelization helps.  \textit{gobmk} is a benchmark which does not experience an improvement for any of these parallelization techniques in BG/Q. This is because no new speculative loops are discovered due to filtering out loops with less than 20\% coverage.  \textit{h264ref} and \textit{sjeng} are two benchmarks where OMP helps but speculative execution negates the performance improvement by OpenMP and results in a slowdown.

\subsubsection{PolyBench/C}

\begin{figure}[h]
\centering
\includegraphics[scale=0.56]{./pdf/poly_O5.pdf}
\caption{Comparison of Speedups gained from different parallelization techniques as compared to the sequential version of the code optimized at the highest level for PolyBench/C Benchmarks}
\label{fig:poly_O5}
\end{figure}

Figure~\ref{fig:poly_O5} gives the effect of different parallelization techniques on the PolyBench/C benchmarks. There is not much improvement from speculative execution for most of the benchmarks other than \textit{gemm} and \textit{lu}. SIMDization and OMP parallelization improves the performance of \textit{gemm} more than 2.5 fold on 4 threads.  For \textit{lu}, the performance improvement is not much promising. \textit{2mm} and \textit{3mm} are benchmarks which performs 2D and 3D matrix multiplication and therefore, TLS is not able to discover new candidate loops for speculation.  SIMD and OMP parallelization achieves good performance improvement already for these two benchmarks as expected. \textit{ jacobi, seidel, cholesky} and \textit{dynprog} experiences slowdown for executing loops speculatively.  The reasons for this slowdown is investigated and explained later.  But this slowdown is not from the execution of colder loops as was seen on the previous experiments with \textit{polly}.  Because the cold loops are already eliminated..  \textit{fdtd-2d} and \textit{gramschimidt} are two benchmarks that do not benefit from any of the three parallelization techniques.

\subsubsection{Investigation of Performance Improvement or Degradation}

The following sets of experiments were performed to investigate the reason of performance improvement or degradation due to different parallelization techniques for the SPEC2006 and PolyBench/C benchmarks.  First the number of loops parallelized and their coverage are calculated.  Next, the misspeculation overhead of the speculatively parallelized versions are investigated to find the reasons of slowdown.  After that, the L1 cache miss rate of the sequential and different parallel versions of the code are calculated using Hardware Performance Monitors (HPM).  Also the instruction path length increase of the parallel versions of the code with respect to the sequential version is calculated.

\subsubsection{Number of loops parallelized and their coverage}

\begin{table}[h]
\centering
\caption{Number of Loops parallelized by different parallel versions of the benchmarks}
\begin{tabular}{|c||c|c|c|c|c|} \hline
Benchmarks & \# Total loops & Speculative & OpenMP & SIMDized & Coverage \\ \hline
lbm & 23 & 5 & 4  &	0 & 98\\ \hline
h264ref & 1870 & 264 & 179 & 3 & 82 \\ \hline
hmmer &	 851	 & 30 & 105 & 17 & 80 \\ \hline
mcf & 52 & 12 & 9 & 0 & 65 \\ \hline
sjeng &	216 & 16	 & 9 & 0 & 32 \\ \hline
sphinx3 & 609 & 2 & 11 & 0 & 91 \\ \hline
bzip2 & 232 & 2 & 4 & 0 & 35 \\ \hline
gobmk & 1265 & 0 & 0 & 0 & - \\ \hline
milc & 421 & 22 & 7 & 2 & 69 \\ \hline
namd & 619 & 25 & 9 & 7 & 92 \\ \hline	
2mm	&20& 7 &7 & 3 & - \\ \hline
3mm	&27& 10&10 & 3 & - \\ \hline	
gemm & 13 & 3 & 4 & 4 & 98 \\ \hline
gramschmidt & 10 & 0 & 3 & 0 & - \\ \hline
jacobi & 9 & 2 & 1 & 0 & 35 \\ \hline
lu & 8 & 5 & 3 & 1 & 45 \\ \hline
seidel & 8 & 2 & 4 & 0 & 40 \\ \hline	
cholesky & 9 & 4 & 0 & 0 & 25 \\ \hline
dynprog & 9 & 3 & 0 & 0 & 26 \\ \hline
fdtd\_2d & 14 & 3 & 0 & 0 & 21 \\ \hline
\hline\end{tabular}
\label{table:coverage_2}
\end{table}

Table~\ref{table:coverage_2} gives the number of parallelized loops by the three different parallelization techniques and their coverage for the SPEC2006 and PolyBench/C benchmarks.  It is clear from the numbers that more number of loops are parallelized for the OMP and speculative parallelization of all these benchmarks than the initial experimentation with \textit{polly}.  The coverage of the parallelized loops are more than 20 percent of the whole program execution time.  The reason for the more number of loops parallelized than the experiments with \textit{polly} is - \textit{polly} only reports loops that are SCoPs.  For being a SCoP, the loop has to have some specific properties that prevents some loops from consideration.  But in the later experiments, the dependence analysis result from LLVM was taken for profiling.  Therefore, the number of loops profiled increased which in turn increased the number of loops reported to be beneficial based on the selection criteria.  \\

The number of loops parallelized by the auto-openmp parallelization in the \textit{bgxlc\_r} compiler is also greater than the number of loops parallelized by \textit{polly} using the polyhedral dependence analysis because \textit{polly} is only considering SCoPs where \textit{bgxlc\_r} is taking all loops (non SCoPs as well) while performing analysis.\\

There is one more relaxation than a conservative analysis for finding the speculation candidates.  Loops with function calls inside them are also considered to be speculatively parallel even if the function call is guaranteed to introduce some dependency inside the loop.  The reason for this relaxation is it is difficult to perform an inter-procedural dependence analysis with the help of profiling.  A more conservative compiler will not consider a loop for speculation if there is a function call inside the loop body and the function is not side-effect free. \\

The interesting benchmark in table~\ref{table:coverage_2} is \textit{h264ref} because it has the most number of speculatively parallelized loops with good coverage than any other but still it experiences a slowdown due to speculative execution.  For \textit{milc} the number of speculative loops is better than most other benchmarks but the loops does not have a high coverage that can account for the speedup from TLS.  Also the coverage of the speculative loops in \textit{sjeng} is on the lower side which adds to the TLS thread creation overhead and contributes to its slowdown.  The number of loops speculated for \textit{lbm} is small but they take most of the execution  time.  These hot loops of \textit{lbm} are good speculation candidates and are examples of cases where TLS can be beneficial.  \textit{namd} also have decent number of speculated loops with a good coverage which contributes to the speed up from TLS.  \textit{bzip2} also has a small number of loops that does not have good coverage, so the speed up achieved for this benchmark from TLS is not from coverage and the number of speculative loops. \textit{gobmk} does not have any parallelizable loops.  This explains the same performance of \textit{gobmk} parallel versions with the sequential version. \\

For PolyBench/C benchmarks, \textit{2mm} and \textit{3mm}, no new speculation candidates were discovered because they are matrix multiplication benchmarks and OpenMP already does a good job parallellizing perfectly parallel loops. For \textit{gemm}, the speculatively parallel loops has a very good coverage that supports the performance improvement from TLS.  For \textit{cholesky}, \textit{dynprog} and \textit{fdtd-2d} the loops paralleized have poor coverage.  This is one of the reasons that accounts for the performance degradation for these benchmarks. \textit{gramschmidt} does not have any new speculative loops discovered because of the filtering criteria of loops with less than 20\% coverage.

\subsubsection{Misspeculation Overhead}

To calculate the misspeculation overhead, the percentage of successfully committed threads are computed for the benchmarks.  The \textit{se\_print\_stats} function from the \textit{speculation\.h} header file is used to collect this statistics.  The \textit{se\_print\_stats} function has to be inserted before and after each speculative loops.  An environment variable can be set to different values to print the statistics after each speculative regions or print the statistics from all the speculative regions at the end of the program.  To percentage of speculative threads committed is computed as -
\begin{equation}
 \frac{speculative }{\left ( speculative + non\_speculative \right )}\ast 100
\end{equation}

\begin{table}[h]
\centering
\caption{The percentage of Speculative Threads successfully committed (not rolled back) for SPEC2006 and PolyBench/C Benchmarks.  It gives the amount of wastage computation. Data is omitted for the benchmarks that has no discovered speculative loop }
\begin{tabular}{|c||c|} \hline
Benchmark &Percentage of Speculative Committed\\ \hline 
lbm &  94\\ \hline
h264ref & 12  \\ \hline
hmmer &	79	  \\ \hline
mcf & 88 \\ \hline
sjeng &	8 \\ \hline
sphinx3 & 29  \\ \hline
bzip2 & 78  \\ \hline
gobmk & -  \\ \hline
milc & 79  \\ \hline
namd & 80  \\ \hline
2mm	& - \\ \hline
3mm	& - \\ \hline
gemm	 & 89 \\ \hline
gramschmidt	& - \\ \hline
jacobi & 78 \\ \hline
lu	& 89 \\ \hline
seidel&	70 \\ \hline
cholesky & 79 \\ \hline
dynprog & 82 \\ \hline
fdtd\_2d & 68 \\ \hline
\hline\end{tabular}
\label{table:spec_committed}
\end{table}

Misspeculation percentage gives the amount of wasted computation for the speculative loops.  Typically a benchmark that can be benefited from TLS should have high percentage of successful completion of speculative threads. Table~\ref{table:spec_committed} gives the percentage of speculative threads committed for the SPEC2006 and PolyBench/C benchmarks.  The best TLS performance is given by \textit{lbm}.  For the two benchmarks \textit{sjeng} and \textit{h264ref} the percentage of speculative threads committed is pretty less giving an indication of a huge amount of wasted computation which resulted in their slowdown.  Closer investigation of these benchmarks reveal that most of the loops speculatively parallelized for these benchmarks contained function calls which resulted in a dependence and thus roll back.  Though \textit{h264ref} has the most number of loops speculatively parallelized, the presence of dependence resulted from function calls inside these loops causes the slowdown for the TLS execution.  Two techniques can be used here to overcome the overhead - 1. The compiler can be conservative and not allow loops with function calls inside them to be executed in parallel regardless of whether the function call changes the dependence behaviour of the loop.  But in that case, we might still miss some opportunities where the function call is harmless.  2. A more sophisticated inter-procedural dependence analysis technique which is able to tell if the function call changes the dependence behaviour of the loop.\\

\textit{mcf}, \textit{hmmer}, \textit{milc}, \textit{namd} and \textit{bzip2} do not have a high percentage of successful speculative execution as \textit{lbm}. This explains the small performance improvement over OpenMP parallelization for these benchmarks as compared to \textit{lbm}. \\

Among the PolyBench/C benchmarks, \textit{gemm} and \textit{lu} has high percentage of speculative thread completion that explains their speedups.  For the four benchmarks - \textit{jacobi}, \textit{seidel}, \textit{cholesky}, \textit{dynprog} there is high percentage of thread completion that does not support the slowdown of these benchmarks.  The reasons of performance degradation for these four benchmarks is investigated further.

\subsubsection{L1 Cache Miss Rate}

The TLS runtime works in close collaboration with the Transactional Memory (TM) runtime of BG/Q.  One of the most dominant BG/Q TM runtime overhead is caused by the loss of L1 cache support due to the L1 flush and bypass needed for the bookkeeping of speculative
state in L2. Though the L2 and L1P buffer load latencies are 13x and 5x higher than the L1 load latency, the L1 miss rate for the sequential ans different parallel versions of the code gives an idea about the performance gain or loss for the benchmarks.\\

The Hardware Performance Monitor (HPM) library of BlueGene/Q is used to collect these statistics. Table~\ref{table:cache} gives the L1 cache misses for the sequential version and the three parallel versions of the SPEC2006 and PolyBench/C benchmarks.\\

\begin{table}[h]
\centering
\caption{\#L1 Cache Hit Rate for SPEC2006 and PolyBench/C benchmarks}
\begin{tabular}{|c||c|c|c|c|} \hline
Benchmark &Sequential & SIMD & AutoOMP & Speculative\\ \hline 
lbm & 95 & 94 & 94 & 93\\ \hline
h264ref & 96 & 95 & 95 & 94 \\ \hline
hmmer &	98 & 97 & 97 & 95  \\ \hline
mcf & 92 & 92 & 95 & 95 \\ \hline
sjeng &	96 & 96 & 95 & 90 \\ \hline
sphinx3 & 96 & 96 & 95 & 95  \\ \hline
bzip2 & 95 & 95 & 95 & 97  \\ \hline
gobmk & 97 & 97 & 97 & 97  \\ \hline
milc & 95 & 97 & 97 & 98  \\ \hline
namd & 96 & 98 & 97 & 98  \\ \hline
2mm	& 98 & 98 & 99 & 99 \\ \hline
3mm	& 98 & 98 & 99 & 99 \\ \hline
gemm	 & 98 & 96 & 98 & 98\\ \hline
gramschmidt	& 97 & 97 & 97 & - \\ \hline
jacobi & 97& 97 & 97 & 97 \\ \hline
lu	& 96 & 96 & 95 & 96 \\ \hline
seidel&	98 & 97 & 98 & 98 \\ \hline
cholesky & 98 & 98 & 96 & 88\\ \hline
dynprog & 97 & 96 & 97 & 90\\ \hline
fdtd\_2d & 98 & 98 & 98 & 98 \\ \hline
\hline\end{tabular}
\label{table:cache}
\end{table}

For \textit{lbm}, the three parallel versions of the code results in more cache misses thus limiting the performance improvement.  But for \textit{mcf}, both OMP and speculative versions result in higher L1 hits which contributes to the performance improvement.  The speculative execution of \textit{sjeng} also results in a high L1 miss rate.  This is the effect of flushing the L1 cache before entering TLS region in the long-running mode.  Apart from the function calls inside TLS regions which introduce data dependences, the high L1 miss rate affects the performance. \\

Similar effect can be seen for the two PolyBench/C benchmarks - \textit{cholesky} and \textit{dynprog}.  Though these two benchmarks have high percentage of successful completion of speculative threads, the speculative execution of the selected loops affects the cache performance due to locality of data between threads.  The cost of bringing the data again after flushing the cache accounts for the slowdown in these benchmarks. \\

For \textit{jacobi} and \textit{seidel} benchmarks, though the speculative execution of the loops result in better cache utilization, the benchmarks experiences a slowdown.  The reason for this slowdown is investigated further.  The two benchmarks \textit{fdtd-2d} and \textit{gobmk} does not show any change in cache utilization for the three parallelization techniques.

\subsubsection{Instruction Path length increase}

AutoOpenMP and speculative parallelization inserts call to OpenMP and TLS runtime functions respectively before and after speculative loops.  Also code needs to be inserted for saving the register values for speculative execution so that the system can be rolled back to a previous state in case of a roll back.  Code growth may also occur for SIMD parallelized version of the code due to insertion of new instructions and also for applying different loop optimizations (e.g loop unrolling) performed with the -qhot option.  The effect of the three parallelization techniques on code growth was observed and the results are given in table~\ref{table:instruction_path}. \\

\begin{table}[h]
\centering
\caption{Instruction Path Length Increase With Respect to Sequential Version for SPEC2006 and PolyBench/C benchmarks}
\begin{tabular}{|c||c|c|c|} \hline
Benchmark & SIMD & AutoOMP & Speculative\\ \hline 
lbm & .03 \% & .25 \% & 26 \%\\ \hline
h264ref & .6 \% & 15 \% & 56 \% \\ \hline
hmmer  & 10 \% & 35 \% & 37 \%  \\ \hline
mcf  & 0 \% & 12 \% & 23 \% \\ \hline
sjeng  & 0 \% & 0 \% & 45 \%  \\ \hline
sphinx3  & 0 \% & 18 \% & 19 \%  \\ \hline
bzip2 & 0 \% & 2 \% & 3 \%  \\ \hline
gobmk  & 0 \% & 0 \% & 0 \%  \\ \hline
milc  & 0.9 \% & 12 \% & 23 \%  \\ \hline
namd & 1 \% & 12 \% & 25 \%  \\ \hline
2mm	& 13 \% & 45 \% & 45 \% \\ \hline
3mm	 & 13 \% & 46 \% & 46 \% \\ \hline
gemm	  & 11\% & 45 & 45 \% \\ \hline
gramschmidt	 & 0 \% & 46 \% & 46\% \\ \hline
jacobi & 0 \% & 95 & 112 \%  \\ \hline
lu	 & 1 \% & 12 \% & 13 \% \\ \hline
seidel&	 0.02 \% & 98 \% & 123 \% \\ \hline
cholesky  & 0 \% & 0 \% & 39 \%\\ \hline
dynprog& 0 \% & 0 \% & 75 \%\\ \hline
fdtd\_2d  & 0 \% & 0 \% & 79 \% \\ \hline
\hline\end{tabular}
\label{table:instruction_path}
\end{table}

As seen in the table, the code growth for PolyBench/C benchmarks is larger than the SPEC2006 benchmarks.  The code growth happens because the loops constitute large part of the PolyBench/C benchmarks and applying optimizations to the effects the code size the most.  For SPEC2006 benchmarks the code growth is pretty small, the highest being for the speculative parallelization of the \textit{h264ref} benchmark because the highest number of loops were speculatively parallelized for this benchmark. \\

\textit{jacobi} and \textit{seidel} from the PolyBench/C suite experiences a massive code growth which explains the reason for their slowdown.  The code growth happens for two reasons - 1. The loops parallelized constitutes the maximum code portion of the benchmark. 2. The memory footprint of the loops speculatively parallelized are large so the register manipulation for speculative execution for the loops adds lots of instructions to the two benchmarks.

\chapter{Conclusion}
\chapter{Related Work}

In this chapter, the previous work on TLS and data dependence profiling used for TLS is given.  There are two separate sections in this chapter - Section 1 summarizes the previous work on TLS while Section 2 gives related work for data dependence profiling.

\section{Thread Level Speculation}

One of the earlier works on TLS is based on Address Resolution Buffers (ARB) by Franklin \textit{et al}~\cite{ARB}.  They propose a hardware mechanism, ARB for performing dynamic reordering of memory references.  ARB has support for the following features - 1) dynamic disambiguation of memory references 2) multiple memory
references per cycle, 3) out-of-order execution of memory references, 4) unresolved loads and stores, 5) speculative loads and stores, and 6) memory renaming. ARB is a shared table that is used to track speculative loads and stores.  The scheme allows speculative loads and speculative stores by keeping the uncommitted store values in the hardware structure and forwarding them to subsequent loads that require the value.\\
After the work by Franklin \textit{et al.}, multiple proposals have been made to move speculative data into each core’s private cache or write buffer, and leverage the cache coherence protocol for memory disambiguation.   In contrast to ARB that uses a centralized buffer to support speculative versions, the work of Vijaykumar \textit{et al.}'s approach, called the Speculative Versioning Cache (SVC), uses distributed caches to eliminate the latency and bandwidth problems of the ARB~\cite{Gopal}. The SVC approach is based on unification of cache coherence and speculative versioning by using an organization similar to snooping bus-based coherent caches. Memory references that gets a hit in the private cache do not use the bus as in an Symmetric Multi-Processor. The committed tasks do not write back the speculative versions all at one time. Instead, versions are marked, together, as committed at commit time, without performing any data movement. Each cache line is individually handled when it is accessed the next time.\\
Lance Hammond \textit{et al.} describes the implementation of speculative execution in a chip multi-processor (CMP), called Hydra~\cite{Hydra}. Their approach a hardware+software hybrid mechanism (a number of software speculation control handlers and modifications to the shared secondary cache memory system of the CMP) to achieve TLS. Their results show that TLS is profitable for applications with substantial amount of medium-grained loops.  When the granularity of parallelism is too small or the available \textit{inherent} parallelism in the application is low, the overhead of the software speculation handlers overwhelms the potential performance benefits from TLS.\\
Steffan \textit{et al.} proposes and evaluates a TLS system that scales to any machine size.  The strategy is a straightforward extension of writeback invalidation-based cache coherence~\cite{steffanISCA00}. The scheme has (i) a notion of whether a cache line has been speculatively loaded and/or speculatively modified; (ii) a guarantee that a speculative cache line will not be propagated to regular memory, and that speculation will fail if a speculative cache line is replaced.  They add three new speculative coherence messages and speculative cache states in the cache coherence protocol.  They show that using their model, applications scale from single-chip multiprocessors or simultaneously multi-threaded processors up to large-scale machines which might use single-chip multiprocessors.\\
Luis Ceze \textit{et al.} describes a an approach to disambiguate memory references so that the dependences in the code can be checked while execution and threads can be committed or rolled back accordingly~\cite{bulk}.  They hash-encode a thread's access information in a signature, and then add operations that efficiently process sets of addresses from the signature.  They employ a Bloom-filter-based compact representation of a thread's access information.  They call their mechanism \textit{Bulk} operation as they operate on a set of addresses.  They show that, despite its simplicity, \textit{Bulk} has competitive performance with more complex schemes.\\
Steffan\textit{ et al.} shows that the performance of a TLS system is dependent on the way threads communicate values among them~\cite{value_communication}.  They apply three different actions - value prediction, dynamic synchronization, and hardware instruction prioritization to improve value communication among threads. Their technique first explores how \textit{silent stores} (value of the memory location before the store is the same as the value of the location after the store) can be exploited within TLS. The TLS system avoids data dependence violations and makes dependent store-load pair independent if the store is silent.  In this way the TLS system reduces the coherence traffic as well as future update traffic. The silent store approach yields similar performance to value predictors but requires less complex design.  They also use some compiler heuristics to select loops for TLS (based on the execution time, number of iterations of the loop etc.) and apply compiler optimization (e.g loop unrolling, reducing critical forwarding path of values between loop iterations etc.) that significantly improves TLS performance.\\
\\
\\
\\

\subsubsection{Loop Selection for Speculative Execution}

Because loops are normally targeted for TLS, there has been work to provide heuristics to select loops for speculative execution.\\
Colohan \textit{et al.} empirically studies the impact of thread size on the performance of loops~\cite{colohan1}. They employ different techniques to unroll loops in order to determine the best thread size per loop. They also propose a runtime system to measure the performance and select each loop dynamically. Due to the runtime overhead, the system can only select loops locally without considering loop nesting.\\
Oplinger \textit{et al.} proposes and evaluates a static loop selection algorithm using simulation~\cite{olukotun}. In their algorithm, they select the best loops in each level of dynamic loop nest as possible candidates to be parallelized and then compute the frequency with which each loop is selected as best loop. Finally, they select the parallelized loops based on the computed frequencies. The concept of dynamic loop nest is similar to the loop tree proposed in our study. However, this technique is only used to guide the heuristic in context-insensitive loop selection. Their performance estimation is obtained directly from simulation, and does not consider the effect of compiler optimization.\\
Chen \textit{et al.} proposes a dynamic loop selection framework for Java program~\cite{chen1}. They use hardware to extract useful information, such as dependence timing, and speculative state requirements and then estimate the speedup for a loop. Their technique is similar to the runtime system proposed by Colohan \textit{et al.} and can only select loops within a simple loop nested program. Considering the global loop nesting relations and selecting the loops globally introduces significant overhead for the run-time system.\\
Marcuello \textit{et al.} proposes a thread spawning scheme that supports spawning threads from any point in the program~\cite{marcuello}. They use profile to identify appropriate thread spawning points with more emphasis on thread predictability.\\
Wang \textit{et al.} proposes different techniques to select loops for TLS~\cite{loopselection}.  They use a construct called loop-graph which describes the different nesting relations among loops.  Only one loop from a loop nest is parallelized based on some heuristics.  They use run-time profile information to measure the performance improvement of parallelizing a loop.  They also apply graph-pruning to reduce the "selection problem" size (which is otherwise NP complete).  Most importantly, they show that the dynamic behaviour of a loop is sensitive to its calling context (for some invocation, the loop may gain performance improvement and for some, it may not).  They attach the calling context information in the loop graph.

\section{Data Dependence Profiling for TLS}

The necessity of data dependence profiling to find TLS candidate loops that cannot be discovered by static dependence analysis is explored in the literature.\\
Ju \textit{et al.} proposed a unified framework to exploit both data and control speculation targeting specifically for memory latency hiding~\cite{ju_spec}. The speculation is exploited by hoisting load instructions across potentially aliasing store instructions or conditional branches.  The framework also contains recovery model which relies on compiler-generated explicit recovery blocks.  Their framework shows modest performance improvement in SPECInt95 benchmarks by shortening critical paths despite some undesirable side effects, such as recovery code penalty during misspeculation and higher register pressures.\\
Chen \textit{et al.} propose a data dependence profiler targeted at speculative optimizations~\cite{chen_spec}. They perform speculative PRE and code scheduling, using a naive profiler and speculative support provided through the ALAT (Advanced Load Address Table) in the Itanium processors. Their non-sampling profiler has a slowdown of up to 100x, and they propose sampling techniques to overcome this problem. They use a shadow space with a simple hashing scheme to facilitate fast address comparison for detecting data dependences.\\
Lin \textit{et al.} propose a compiler framework for speculative analysis and optimizations~\cite{lin_spec}.  The framework is based on a speculative SSA (Static Single Assignment) form to incorporate speculative information for both data and control speculation.  Speculative SSA integrates the alias information directly into the intermediate representation using explicit \textit{may modify operator} and \textit{may reference operator}. Their speculative analysis is assisted by both \textit{alias profiles} and heuristic rules.\\
The Multiscalar compiler selects tasks by walking the Control Flow Graph (CFG) and accu-
mulating basic blocks into tasks using a variety of heuristics~\cite{multiscalar}. The task selection methodology for the Multiscalar compiler was revisited by Johnson \textit{et al}~\cite{mincut}.  Instead of using a heuristic to collect basic blocks into tasks, the CFG is now annotated with weights and broken into tasks using a min-cut algorithm. These compilers assume special hardware for dispatching threads; they do not specify when a thread should be launched.\\
In SPSM, loop iterations are selected by the compiler as speculative threads~\cite{SPSM}. They use a special instruction \textit{fork} that allows the compiler to specify when tasks begin executing. In addition, SPSM recognizes the potential benefits from prefetching but proposes no techniques to exploit it. \\
Bhowmik and Franklin build a framework for speculative multithreading on the SUIF-MachSUIF platform~\cite{bhowmik}. Within this framework, they consider dependence-based task selection algorithms. Like Multiscalar, they focus on compiling the whole program for speculation, but allow the compiler to specify a spawn
location as in SPSM.\\
Du et al. propose a cost-driven compilation framework to perform speculative parallelization~\cite{du_spec}. The compiler uses dependence profile for task selection and for partitioning speculative loops into serial and parallel portions. The profiler tracks both intra- and carried- true dependences for speculative loops. Carried-dependences are used to guide the parition of loop bodies into a serial and a parallel portion. Since dependences originated from the serial portion do not trigger roll-back in the parallel portion, the key part of their framework is to move source computation of frequent dependences (called violating candidate) to the serial portion through instruction reording. A cost model is used to select the optimal loop partition, which is based on the size of serial portion and the misspeculation cost of the parallel portion. The latter is computed by combining re-execution cost of individual nodes weighted by probabilities of carried-dependences (for violating candidates) and intra-iteration dependences (for others).  They achieve a 8\% speedup for the SPECInt2000 benchmarks.\\
Quinones \textit{et al.} evaluated the Mitosis compiler for exploiting speculative thread-level parallelism~\cite{mitosis}.  Mitosis compiler uses both dependence and edge-profiles for 1)generating speculative precomputation slices (p-slice) and 2) selecting spawning pairs. P-slice predicts live-in values for speculative tasks and contributes to the serial portion of the speculative execution. To minimize p-slice overhead while maximizing the accuracy, the compiler uses dependence- and edge-profiles to prune instructions in p-slices. To select spawning pairs, another profile analyzes the sequential execution trace to model the speculative execution time of each candidate spawning pair without considering inter-task memory conflicts. Instead, in this execution model, task squashes is mostly determined by mispredication probability of p-slices. They get a 2.2x speedup for the OLDEN benchmarks on their simulator.\\
Wu \textit{et al.} propose a cost model for compiler-driven task selection for TLS~\cite{dprof}.  The model employs profile-based analysis of may-dependences to estimate the probability of successful speculation.  Their profiling tool, \textit{DProf}, is able to  measure dependence probability and independence window of loops. They provide two techniques - dependence windows (the number of iterations that can be executed in parallel without affecting the output) and dependence clustering (regions of the loop's iteration space with large independence windows that makes that portion profitable to execute in parallel).  One improtant finding is that there is little variability in independence window width in the hot loops for the SPEC2000 benchmarks. Loops are either parallel or serial with an independence window width of 1.\\
Kim \textit{et al.} propose a scalable approach to data-dependence profiling that addresses both runtime and memory overhead in a single framework~\cite{sd3}.  Their technique, called \textit{SD3}, reduces the runtime overhead by parallelizing the dependence profiling step itself. To reduce the memory overhead, they compress memory accesses that exhibit stride patterns and compute data dependences directly in a compressed format. For stride detection, they use a \textit{finite state machine} (FSM).SD3 reduces the runtime overhead by a factor of  9.7× on  32 cores and reduces memory consumption by 20x with a 16x speedup in profiling time.  They also show that the dependence behaviour for different inputs \textit{does not} change for OmpSCR benchmarks and changes very less (correlation 0.98 among the dependence pairs discovered in the loops) for SPEC2006 benchmarks.  This finding is important and supports the findings of my research.\\
POSH is a TLS compiler built on top of gcc~\cite{posh}. POSH performs a partitioning of the code into TLS tasks and considers both subroutines and loops. POSH also uses a simple profiling pass to discard ineffective tasks.  While choosing tasks beneficial for speculation, POSH considers the task size and the squash frequency which is obtained by simulation of the parallel execution.  L2 cache misses are also considered by the profiler to perform prefetching analysis.\\
Vanka \textit{et al.} approaches data dependence profiling from a set-based approach~\cite{vanka}.  Rather than tracking pair-wise dependences, they identify important dependence relationships at compile time, group said relationships into sets, and then construct and operate directly on the sets at runtime.  They use \textit{Signatures}, an efficient set representation based on Bloom filters, to to trade-off performance and accuracy.\\
None of these approaches provide a solution for loops which has varied dependence behaviour for different executions.


%\include{Chapters/introduction}



%
%\include{Chapters/related-work}
%
%\include{Chapters/contrib1}
%
%\include{Chapters/contrib2}
%
%\include{Chapters/experiments}
%
%\include{Chapters/conclusions}

\clearpage\addcontentsline{toc}{chapter}{Bibliography}
     %add the above line to get "Bibliography" in the table of contents.
%
\singlespacing % optional;  Bibliography is better in single spacing
               %            but you may choose different
               %            Don't use \singlespacing if your thesis
               %            is already in single spacing
%
\bibliographystyle{plain} % Or which ever you wish. Plain is good
                          % for long bibs.
\bibliography{arnamoy}

\doublespacing

\appendix  %  If you have any appendices
            % Use standard Latex sectioning commands
            % like \chapter etc.

%\include{Chapters/app1}

\end{document}
